\begin{DoxyAuthor}{Authors}
Daniel Hong 

Saba Shaikh 

Nnamdi Ibe 

Ali El-\/\+Farou 

Khalid Awil 

Ogba Okparanyote 
\end{DoxyAuthor}
\begin{DoxyDate}{Date}
2016-\/02-\/07 
\end{DoxyDate}
\section{Introduction}\label{index_Introduction}
\begin{DoxyParagraph}{}
This document describes the system requirements for Sprint 1. The system design is divided into hardware and software design. Hardware design describes the hardware used in the current product including background information to help understand the design. The software design provides background information, overview information of other software components used in the current product (in particular free\+R\+T\+OS and its components),and a detailed design of the current product. Software design is modular and uses C software modules. ~\newline

\end{DoxyParagraph}
\subsection{1.\+1 Scope}\label{index_subsection_name_1_1}
\begin{DoxyParagraph}{}
This software product is identified as Chico\+: The Robot version 1.\+0.\+0. The expected functionalities and deliverables are\+:
\begin{DoxyItemize}
\item Setup and test the development environment, document background information about the Stingray robot and various components, the Arduino Mega 2560 card, and the Free\+R\+T\+OS operating system
\item Create a first version of the Product Release Document that presents the product requirements and design (this document)
\item Complete the first release of the Chico\+: The Robot software, which will deliver functionalities and simulate the atmospheric temperature monitor overall. The rest of the document covers the overall description of the product and expected functionalities. The next section describes detailed requirements and deliverables for Sprint 1. See the references section for a list of reference documents that describe the details on the hardware and software components. ~\newline
 
\end{DoxyItemize}
\end{DoxyParagraph}
\begin{DoxyParagraph}{}

\end{DoxyParagraph}
\section{2 Overall Description}\label{index_section_name_1}
The Stingray Wi\+Fi robot is equipped with various hardware components such as the Arduino Mega 2560 microcontroller, a Serial L\+CD, Wi-\/\+Fi, an Ultrasonic sonar sensor, servo motors, and a photo reflector circuit. This section provides some details of these components to help the user understand the environment. \subsection{2.\+1 Product Perspective}\label{index_subsection_name_2_1}
\begin{DoxyParagraph}{}
The Stingray Wi\+Fi robot currently mimics the functionality of a temperature monitor. This is achieved by using the existing hardware, including the Arduino Mega 2560 microcontroller, the Sparkfun H\+D44780U controller based serial 2x16 L\+CD, the Sandbox Hydrogen Wi\+Fi shield with built-\/in webserver, and the T\+P\+A81 Thermopile Array (Thermal Array Sensor) along with the Free\+R\+T\+OS software modules to manipulate these components. Refer to the references section for more details on each hardware component. 
\end{DoxyParagraph}
\subsection{2.\+2 Hardware Componenets}\label{index_subsection_name_2_2}
\subsubsection{Stingray Wifi Robot}\label{index_subsubsection_name_2_2_1}
\begin{DoxyParagraph}{}
The Stingray Wifi Robot (figure 1) specific components include\+:
\begin{DoxyItemize}
\item 2 Futuba F\+P-\/\+S148 continuous rotation servo motors for differential steering
\item Hamamatsu P5587 photo reflector circuit with wheel incremental encoders
\item Parallax P\+I\+NG ultrasonic sensor on a rotating motor
\item 8 pixel thermal array sensor
\item (D\+IY Sandbox) Arduino Hydrogen Wi\+Fi with built in web server
\item Serial 16x2 L\+CD
\item Arduino Mega 2560 Microcontroller
\item 2 Ni\+Cad 1700 m\+A/h batteries and AC adaptor  The Arduino Mega2560 (figure 2) is a microcontroller board based on the A\+Tmega2560. The A\+VR 2560-\/16\+AU installed on the Arduino Mega 2560 card is the 100 pin version which provides all parts shown in figure 3. It provides 256 Kbytes of flash memory, 4 Kbytes of E\+E\+P\+R\+OM, 8 Kbytes of R\+AM, 86 general purpose I/O pins, 12 16-\/bit resolution P\+WM channels, serial U\+S\+A\+R\+TS, and 16 A\+DC channels. Refer to the A\+VR 2560 Micro\+Controller/\+Arduino/\+Sting\+Ray Summary Documentation for more details on the port functions, mapping, and memory map.   The main hardware components used for Sprint 1 include the serial L\+CD module, 8 pixel thermal array sensor, and the Hydrogen Wi\+Fi L\+E\+Ds. ~\newline

\end{DoxyItemize}
\end{DoxyParagraph}
\subsubsection{Stingray Robot Arduino Pinout}\label{index_subsubsection_name_2_2_2}
This section lists the Arduino Mega pins and its associated hardware of the Stingray robot. See the references section for more details on each part.

Wi-\/\+Fi Hydrogen Board


\begin{DoxyItemize}
\item Communication 6 -\/ T\+X2
\item Communication 5 -\/\+R\+X2
\item P\+WM 10 -\/ Sd Card Cs
\item P\+WM 11 -\/ Sd card M\+O\+SI
\item P\+WM 12 -\/ Sd Card M\+I\+SO
\item P\+WM 13 -\/ Sd Card C\+LK
\item P\+WM 3 -\/ Blue L\+ED
\item P\+WM 5 -\/ Green L\+ED
\item P\+WM 6 -\/ Red L\+ED
\item Reset -\/ Reset
\end{DoxyItemize}

Servos
\begin{DoxyItemize}
\item P\+WM 4 -\/ right servo motor
\item P\+WM 2 -\/ left Servo motor
\item P\+WM 7 -\/ center servo motor
\end{DoxyItemize}

Sensors and Display
\begin{DoxyItemize}
\item Digital I/O 22 -\/ Sonar Input
\item Digital I/O 26 -\/ Left Encoder input
\item Digital I/O 28 -\/ Right Encoder Input
\item Serial T\+X1 -\/ L\+CD Display
\item S\+DA 20 -\/ I2C Bus
\item S\+CL 21-\/ I2C Clock Bus ~\newline

\end{DoxyItemize}\subsubsection{Serial L\+C\+D Module}\label{index_subsubsection_name_2_2_3}
 \begin{DoxyParagraph}{}
The 16\+X2 Liquid Crystal display (L\+CD), shown in figure 4, is used to display characters. A ser\+L\+CD v2.\+5 module is used to interface the L\+CD. The module takes incoming 9600bps T\+TL levels and displays the signals on the L\+CD. Only three wires(5\+V, G\+N\+D and Signal) are used to interface the L\+CD. 
\end{DoxyParagraph}
\begin{DoxyParagraph}{}
Ser\+L\+CD Features
\begin{DoxyItemize}
\item Incoming buffer stores up to 80 characters
\item All surface mount design allows a backpack that is half the size of the of the original
\item Faster boot up time than previous versions
\item User definable splash screen included
\item Processing speeds at 8\+M\+HZ
\item Boot up display can be turned either on or off via firmware
\item Adjustable baud rates between 2400,4800,9600,14400,19200 and 38400
\item Operational backspace
\item Backlight supports pulse width modulation
\item The backlight transistor can handle up to 1A
\item New P\+IC 16\+F688 utilizes onboard U\+A\+RT for greater communication accuracy 
\end{DoxyItemize}
\end{DoxyParagraph}
\subsubsection{Thermal Array Sensor}\label{index_subsubsection_name_2_2_4}
 \begin{DoxyParagraph}{}
The T\+P\+A81 (see figure 5) is a thermopile array detecting infra-\/red in the range of 2um-\/22um. This is the wavelength of radiant heat. It has an array of eight thermopiles arranged in a row. There are 9 temperature readings available, all in degrees centigrade (�C). Register 1 is the ambient temperature as measured within the sensor. The T\+P\+A81 has 10 registers. Register 0 is the command register, registers 2-\/9 are the 8 pixel temperatures. Temperature acquisition is continuously performed and the readings will be correct within approximately 40ms after the sensor points to a new position. 
\end{DoxyParagraph}
\begin{DoxyParagraph}{Features}

\begin{DoxyItemize}
\item Can detect a candle flame from 6ft
\item Unaffected by ambient light
\item Simultaneously measures temperature of 8 adjacent points
\item Can control a servo to pan the module and build a thermal image
\item All communications are done via I2C.  
\end{DoxyItemize}
\end{DoxyParagraph}
\subsubsection{Wi-\/\+Fi Hydrogen Shield}\label{index_subsubsection_name_2_2_5}
The Hydrogen from D\+IY Sandbox (see figure 7) is a fully integrated Wi\+Fi Arduino shield. Hydrogen provides certified 802.\+11b Wi\+Fi connectivity along with an integrated micro\+SD card, true R\+GB P\+WM L\+E\+Ds, integrated 3.\+3V L\+DO, dual stackable connectors, and hardware/software U\+A\+RT with S\+PI chip select.  \begin{DoxyParagraph}{}
Features
\begin{DoxyItemize}
\item 802.\+11b certified Wi\+Fi Module by Gain\+Span or integrated antenna
\item 11 Mbps Wi\+Fi module (throughput limited by Arduino processor clock speed)
\item Universal connectivity to adhoc and infrastructure networks
\item U\+F.\+L connector for external antenna
\item Switchable software or hardware U\+A\+RT connection o W\+E\+P64/128,W\+P\+A/\+W\+P\+A2
\item Supports Wi\+Fi access point generation
\item Supports up to 16 T\+CP or U\+DP connections
\item Dynamic site scanning of nearby access points including R\+S\+SI information or D\+H\+CP \& D\+NS lookup
\item Micro SD Card Holder
\item Three Pulse Width Modulated (P\+WM) L\+E\+Ds
\item Individual on/off pin and fully customizable power levels
\item Red Blue and Green L\+E\+Ds for brilliant color combinations  
\end{DoxyItemize}
\end{DoxyParagraph}
\subsection{2.\+3 Product Functions}\label{index_subsection_name_2_3}
\begin{DoxyParagraph}{}
The objective of the Stingray Wi\+Fi robot, was to act as a temperature monitor. This was accomplished with three main functions, which work hand in hand to deliver the objective.
\begin{DoxyItemize}
\item Reading the temperature from the thermal array sensor
\item Indicating the average temperature on the L\+ED by change of color
\item Displaying the ambient + 8 pixel temperatures on the L\+CD 
\end{DoxyItemize}
\end{DoxyParagraph}
\begin{DoxyParagraph}{}
Firstly, the temperature is read from the thermal array sensor. This is accomplished by using the I2C Software Protocol to establish a communication between the microcontroller and the thermal array sensor. The microcontroller (acting as the master device), goes on to read the ambient temperature of the thermal sensor, as well as the surrounding 8 pixel temperatures; after which they are then made available to the other modules. 
\end{DoxyParagraph}
\begin{DoxyParagraph}{}
Given the average temperature collected from the thermal sensors; the Hydrogen Wi\+Fi shield L\+ED is used the indicate its range. The color of the L\+ED is either blue, green or red, indicating the average temperature is below 30oC, between 30 and 40oC or 40oC or greater respectively. The microcontroller is used to turn on the L\+ED using I/O voltage on pins, by sending a 0V. The intensity of the L\+ED is also controlled using the I/O voltage pin, through pulse width modulation. 
\end{DoxyParagraph}
\begin{DoxyParagraph}{}
The L\+CD display is used to display the ambient temperature of the thermal sensor, as well as the 8 pixel temperatures. 
\end{DoxyParagraph}
\section{3 Requirements}\label{index_section_name_3}
\subsection{3.\+1 Software Requirements}\label{index_subsection_name_3_1}
\begin{DoxyParagraph}{}
This section provides detailed description of the functional requirements of Sprint 1. The requirements have been broken down into sections for each hardware component of the system. 
\end{DoxyParagraph}
\begin{DoxyParagraph}{}
Temperature sensor
\begin{DoxyItemize}
\item The system shall read the surrounding ambient temperature using the thermal array sensor.
\item The system shall read the 8 pixel temperatures using the thermal array sensor
\item The system shall switch on the L\+E\+Ds to emit three specific colours according to the average surrounding temperature
\begin{DoxyItemize}
\item The system shall emit a blue light using the L\+E\+Ds if the average temperature is below 30 degrees celsius.
\item The system shall emit a green light if the average temperature ranges between 30 and 40 degrees celsius.
\end{DoxyItemize}
\item The system shall emit a red light if the temperature equals or exceeds 40 degrees celsius. 
\end{DoxyItemize}
\end{DoxyParagraph}
\begin{DoxyParagraph}{}
L\+CD Screen
\begin{DoxyItemize}
\item The system shall display temperatures represented by at most two digits
\item The system shall use the functions from usartserial.\+h as the bases of sending commands and writing to the L\+CD screen
\item The system shall open the appropriate U\+S\+A\+RT serial port for output to the L\+CD
\begin{DoxyItemize}
\item The appropriate baud rate shall be set (9600)
\item U\+S\+A\+R\+T1\+\_\+\+ID shall be the U\+S\+A\+RT serial port to open
\item Queue lengths shall be set to constants port\+S\+E\+R\+I\+A\+L\+\_\+\+B\+U\+F\+F\+E\+R\+\_\+\+TX and port\+S\+E\+R\+I\+A\+L\+\_\+\+B\+U\+F\+F\+E\+R\+\_\+\+RX respectively
\end{DoxyItemize}
\item The system shall display \textquotesingle{}A\+M\+BT\+: \textquotesingle{}to indicate the digits that represent the surrounding ambient temperature.
\item The system shall display the surrounding ambient temperature on 1 line of the L\+CD screen
\item The system shall display an appropriate message on the L\+CD screen when displaying the 8 pixel temperatures.
\item The system shall clear the screen before each display update
\item The system shall send command codes to the L\+CD to display on the appropriate lines
\item The system shall display the temperature for 1000 / port\+T\+I\+C\+K\+\_\+\+P\+E\+R\+I\+O\+D\+\_\+\+MS before updating 
\end{DoxyItemize}
\end{DoxyParagraph}
\section{Software Design}\label{index_section_name_4}
\subsection{4.\+1 Background Information}\label{index_subsection_name_4_1}
 \begin{DoxyParagraph}{}
To enable reading from and writing to the thermal array sensor, the I2\+C\+Multi\+Master software module was used. This module allows I2C communication via the hardware module provided in the Atmega2560 controller; thereby enabling us to configure the Atmega controller as the master device and the thermal array sensor as the slave device.
\end{DoxyParagraph}
To print information to the L\+CD, the C\+E\+G4166\+\_\+\+R\+T\+S\+\_\+\+L\+IB U\+S\+A\+RT module was used, this module allowed us to open the usart port for output to the L\+CD; and thereby enabling us to print characters on the L\+CD screen. \subsection{4.\+2 Thermal Sensor Software Module}\label{index_subsection_name_4_2}
\begin{DoxyParagraph}{}
This module is responsible for reading information from the thermal array sensor. Communication between the atmega controller and the thermal sensor is done by using the functions provided by the I2\+C\+Multi\+Master module. 
\end{DoxyParagraph}
\begin{DoxyParagraph}{}
Firstly, the start sequence is sent with the master address, using the I2\+C\+\_\+\+Master\+\_\+\+Initialise function. Then the address of the thermal sensor (the slave device), along with the address of the internal address to read from, are then sent using the I2\+C\+\_\+\+Master\+\_\+\+Start\+\_\+\+Transceiver\+\_\+\+With\+\_\+\+Data function. Afterwards, the read command is sent to the slave device with the I2\+C\+\_\+\+Master\+\_\+\+Start\+\_\+\+Transceiver\+\_\+\+With\+\_\+\+Data funciton, and then the I2\+C\+\_\+\+Master\+\_\+\+Get\+\_\+\+Data\+\_\+\+From\+\_\+\+Transceiver function is used to gather the data gotten the thermal sensor. 
\end{DoxyParagraph}
\subsection{4.\+3 Serial\+L\+C\+D\+Module Software Module}\label{index_subsection_name_4_3}
\begin{DoxyParagraph}{}
This module is used to print characters to the screen. Therefore, this module is responsible for printing the ambient temperature, as well as the 8 pixel temperatures read from thermal array sensor.
\end{DoxyParagraph}
The screen is first cleared, and then using the usart\+Write function, the cursor is positioned on the first spot on the L\+CD, in order to print the message �\+A\+M\+BT\+:�. Following this, the ambient temperature along with the 8 pixel temperatures are printed on the screen, moving the cursor each time to allow spacing between the temperature readings. After printing each set of the temperature values, a delay of 1000/port\+T\+I\+C\+K\+\_\+period\+\_\+\+MS is used. This is done in order to keep temperature reading on the screen long enough to read. \subsection{4.\+4 L\+E\+D Software Module}\label{index_subsection_name_4_4}
\begin{DoxyParagraph}{}
This module is used to change the color of the L\+ED, depending on the average temperature from the thermal sensor. Firstly, the pins assigned to the three L\+ED colors (blue, green, red) are set to output at 5V. Afterwards, the L\+ED colors are set according to the average value using the avr/io modules. 
\end{DoxyParagraph}
\section{5 References}\label{index_section_name_5}
\begin{DoxyParagraph}{}
G. Arbez, �\+Sting\+Ray Wi\+Fi Robot Documentation C\+S\+I4166/\+C\+S\+I4141,� rep., 2015. 
\end{DoxyParagraph}
\begin{DoxyParagraph}{}
G. Arbez, �\+A\+VR 2560 Micro\+Controller/\+Arduino/\+Sting\+Ray Summary Documentation C\+E\+G4166/\+C\+S\+I4141,� rep., 2015. 
\end{DoxyParagraph}
\begin{DoxyParagraph}{}
S., �\+Sparkfun Electronics Ser\+L\+CD v2.\+5 Serial Enabled L\+CD,� 2006. 
\end{DoxyParagraph}
\begin{DoxyParagraph}{}
�\+T\+P\+A81 Infra Red Thermal Sensor,� T\+P\+A81 Infra Red Thermal Sensor. [Online]. Available at\+: {\tt http\+://www.\+robot-\/electronics.\+co.\+uk/htm/tpa81tech.\+htm}. [Accessed\+: Feb-\/2016].
\end{DoxyParagraph}


 